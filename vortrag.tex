\documentclass{beamer}
\usepackage[utf8]{inputenc}
\usepackage[ngerman]{babel}
\usepackage{tabularx,ragged2e}
\usepackage{xspace}
%\usepackage{href}

\usepackage{graphicx}
\graphicspath{{./img/}}

\usetheme{Frankfurt}  %% Themenwahl

\usecolortheme{seagull}


\title{Einstieg in die Akkreditierung}
\author{Daniela Kern-Michler, Philipp Jaeger}
\date{\today}

\def\bingo{
\footnotesize
  \begin{tabular}{|m{.2\textwidth}|m{.2\textwidth}|m{.2\textwidth}|m{.2\textwidth}|}
  %\begin{tabular}
  \hline
  Akkreditierungs\-verfahren & Hochschul\-rektorenkonferenz (HRK) & System\-akkreditierung & Akkreditierungs\-kriterien \\\hline
  Akkreditierungs\-entscheidung & L\"andergemein\-same Struktur\-vorgaben & Gutachter & Begehung \\\hline
  Akkreditierungs\-agentur & Akkreditierungs\-rat (AR) & Modul & European Credit Transfer and Accumulation System (ECTS) \\\hline
  Modulabschluss\-pr\"ufung & Pr\"ufungs\-leistung & Studien\-leistung & Kultusminister\-konferenz (KMK) \\\hline
  Studienakkredi\-tierungsstaats\-vertrag\newline (StAkkrStV) & Musterrechts\-verordnung (MRVO) &  Gutachten & Programm\-akkreditierung \\\hline
  \end{tabular}
}

\begin{document}
\begin{frame}
\maketitle
\vfill
\includegraphics[scale=0.6]{pool.PNG}
\hfill\includegraphics[width=2cm]{CC-BY-SA.png}
\end{frame}


\begin{frame}
\tableofcontents
\end{frame}

\section{Akkreditierung? Akkreditierung!}

\frame{\tableofcontents[currentsection]}
%%%%%%%%%%%%%%%%%%%%%%%%%%%%%%%%%%%%%%%%%%%%%%%%%%%%%%%%%%%%%%%
\begin{frame}
  \frametitle{Was ist Akkreditierung?}
  \begin{itemize}
  	\item Prüfverfahren für Studiengänge
  	\item eine Art ``TÜV`` oder Qualitätssiegel
  	\item legt Mindeststandards fest
  	\item prüft Studierbarkeit
  	\vspace{1cm}
  	\item mittlerweile (und immer mehr) auch für Qualitätssicherungssysteme $\rightarrow$ Systemakkreditierung
  \end{itemize}   
\end{frame}
%%%%%%%%%%%%%%%%%%%%%%%%%%%%%%%%%%%%%%%%%%%%%%%%%%%%%%%%%%%%%%%
\begin{frame} 
  \frametitle{Historische Schlagwörter zur deutschen Entwicklung} 
  \begin{itemize}
  	\item Bologna-Erklärung 1999
  		\begin{itemize}
  			\item Harmonisierung des europäischen Hochschulsystems
  			\item Gemeinsame Qualitätssicherung
  		\end{itemize}
  	\pause
  	\item Regel zur Akkreditierung und Ländergemeinsame Strukturvorgaben
  	\item Studentenproteste/Bildungsstreiks führen zu Überarbeitungen
  	\item Beschluss des Bundesverfassungsgerichtes erklärt Regelungen in NRW für ungültig (2016)
  	\vspace{1cm}
  	\pause
  	\item \textbf{seit 1.1.2018: Studienakkreditierungsstaatsvertrag und Musterrechtsverordnung zur Akkreditierung sind rechtliche Grundlage}
  	\item Bundesländer erlassen Rechtsverordnungen
  \end{itemize}   
\end{frame}
%%%%%%%%%%%%%%%%%%%%%%%%%%%%%%%%%%%%%%%%%%%%%%%%%%%%%%%%%%%%%%%
\begin{frame} 
  \frametitle{Warum beschäftigen wir uns damit?} 
  \begin{itemize}
  	\item unsere Studiengänge durchlaufen die Verfahren
  		\begin{itemize}
  			\item Studierbarkeit wird zertifiziert
  			\item Aufmerksamkeit für Fragen zu Studium und Lehre
  		\end{itemize}
  	\pause
  	\item wir bringen die studentische Perspektive in die Prüfverfahren $\rightarrow$ Studentischer Akkreditierungspool
  	\pause
  	\item wir nehmen in Resos Stellung zum Akkreditierungssystem und seinen Veränderungen
  \end{itemize}   
\end{frame}
%%%%%%%%%%%%%%%%%%%%%%%%%%%%%%%%%%%%%%%%%%%%%%%%%%%%%%%%%%%%%%%
\begin{frame} 
  \frametitle{Akkreditierungsbingo} 
  \bingo
  %\includegraphics[width=1\textwidth]{akkreditierungs-bingo.png}
%\begin{tabular}{|p{0.22\textwidth}|p{0.22\textwidth}|p{0.22\textwidth}|p{0.22\textwidth}|}
%\hline
%Akkreditierungs-verfahren & Programm-akkreditierung & System-akkreditierung & Akkreditierungs-kriterien\\
%\hline
%Akkreditierungs-entscheidung & Gutachten & Gutachter & Begehung \\
%\hline
%Akkreditierungs-agentur & Akkreditierungs-rat & Modul & European Credit Transfer and Accumulation System (ECTS)\\
%\hline
%Modulabschluss-prüfung & Studienleistung & Prüfungsleistung & Kultusminister-konferenz (KMK)\\
%\hline
%Staatsvertrag & Musterrechts-verordnung & Länder-gemeinsame Strukturvorgaben & Hochschul-rektoren-konferenz (HRK)\\
%\hline
%\end{tabular}
\end{frame}
%%%%%%%%%%%%%%%%%%%%%%%%%%%%%%%%%%%%%%%%%%%%%%%%%%%%%%%%%%%%%%%
\section{Stationenarbeit}
\frame{\tableofcontents[currentsection]}
%%%%%%%%%%%%%%%%%%%%%%%%%%%%%%%%%%%%%%%%%%%%%%%%%%%%%%%%%%%%%%%
\begin{frame} 
  \frametitle{Ablauf Stationenarbeit} 
  \begin{itemize}
  	\item Kleingruppen (3-5 Leute)
  	\item 15 min pro Station (teilweise etwas knapp $\rightarrow$ Überblick gewinnen
  	\vspace{0.5 cm}
  	\item Stationen:
  	\begin{itemize}
  		\item Akkreditierungssystem
  		\item Akkreditierungsverfahren
  		\item Akkreditierungskriterien
  		\item Regeln für die Studiengangsgestaltung
  	\end{itemize}
  \end{itemize}   
\end{frame}
%%%%%%%%%%%%%%%%%%%%%%%%%%%%%%%%%%%%%%%%%%%%%%%%%%%%%%%%%%%%%%%
\begin{frame} 
  \frametitle{Akkreditierungsbingo} 
  %\includegraphics[width=1\textwidth]{akkreditierungs-bingo.png}
  \bingo
\end{frame}
%%%%%%%%%%%%%%%%%%%%%%%%%%%%%%%%%%%%%%%%%%%%%%%%%%%%%%%%%%%%%%%
\section{Nachbesprechung}
\frame{\tableofcontents[currentsection]}
%%%%%%%%%%%%%%%%%%%%%%%%%%%%%%%%%%%%%%%%%%%%%%%%%%%%%%%%%%%%%%%
\begin{frame}
\frametitle{Persönlicher Merkzettel}
Was sind die entscheidenden Dokumente für die Akkreditierung? Wo findet man Informationen?
\vspace{0.5cm}
\begin{itemize}
\item Staatsvertrag
\item Musterrechtsverordnung
\item Landesgesetze
\end{itemize}
\pause
\begin{itemize}
\item Website Akkreditierungsrat%\\\url{https://akkreditierungsrat.de}
\item Website Studentischer Akkreditierungspool%\\\url{https://studentischer-pool.de}
\end{itemize}
\pause
\begin{itemize}
\item Nationaler Qualifikationsrahmen (DQR)
\item Europäische Qualifikationsrahmen (EQR)
\item The Standards and guidelines for quality assurance in the European Higher Education Area (ESG)
\end{itemize}
\end{frame}
%%%%%%%%%%%%%%%%%%%%%%%%%%%%%%%%%%%%%%%%%%%%%%%%%%%%%%%%%%%%%%%
\begin{frame}
\frametitle{Persönlicher Merkzettel}
Wie ist das deutsche Akkreditierungssystem aufgebaut?
\vspace{0.5cm}
  \includegraphics[width=1\textwidth]{Schaubild2.png}
\end{frame}
%%%%%%%%%%%%%%%%%%%%%%%%%%%%%%%%%%%%%%%%%%%%%%%%%%%%%%%%%%%%%%%
\begin{frame}
\frametitle{Persönlicher Merkzettel}
Wie ist das schweizer Akkreditierungssystem aufgebaut?
\begin{itemize}
\item Institutionelle Akkreditierung
\item Programmakkreditierung auf Wunsch
\item nur eine Agentur AAQ
\item Akkreditierungsentscheide durch Akkreditierungsrat
\item Hochschulförderungs- und Koordinationsgesetz (HFKG)
\end{itemize}
\pause
Wie ist das österreichische Akkreditierungssystem aufgebaut?
\begin{itemize}
\item Institutionelle Akkreditierung
\item nur eine Agentur AQA
\item Hochschul-Qualitätsicherungsgesetz (HS-QSG)
\end{itemize}
\end{frame}
%%%%%%%%%%%%%%%%%%%%%%%%%%%%%%%%%%%%%%%%%%%%%%%%%%%%%%%%%%%%%%%
\begin{frame}
\frametitle{Persönlicher Merkzettel}
Wie sieht ein Akkreditierungsverfahren aus?
\vspace{0.5cm}
  \includegraphics[width=1\textwidth]{verfahren.png}\\
Begutachtung muss nicht Vor-Ort Begehung bedeuten...
\end{frame}
%%%%%%%%%%%%%%%%%%%%%%%%%%%%%%%%%%%%%%%%%%%%%%%%%%%%%%%%%%%%%%%
\begin{frame}
\frametitle{Persönlicher Merkzettel}
Was wird geprüft?
\vspace{0.5cm}
\textbf{Formale Kriterien}
\begin{itemize}
\item Studienstruktur und -dauer
\item Studiengangsprofil
\item Zugangsvoraussetzungen und Übergänge
\item Abschlüsse und deren Bezeichnungen
\item Modularisierung
\item Leistungspunktesystem
\item ...
\end{itemize}
\end{frame}
%%%%%%%%%%%%%%%%%%%%%%%%%%%%%%%%%%%%%%%%%%%%%%%%%%%%%%%%%%%%%%%
\begin{frame}
\frametitle{Persönlicher Merkzettel}
Was wird geprüft?
\vspace{0.5cm}
\textbf{Fachlich-inhaltliche Kriterien}
\begin{itemize}
\item Qualifikationsziele und Abschlussniveau
\item Studiengangskonzept und Umsetzung
\item Fachlich-inhaltliche Gestaltung
\item Studienerfolg
\item Geschlechtergerechtigkeit und Nachteilsausgleich
\item ...
\end{itemize}
\end{frame}
%%%%%%%%%%%%%%%%%%%%%%%%%%%%%%%%%%%%%%%%%%%%%%%%%%%%%%%%%%%%%%%
\begin{frame}
\frametitle{Persönlicher Merkzettel}
Was für Akkreditierungsentscheidungen gibt es? Wie lange gelten sie?
\begin{itemize}
\item Akkreditierung und Nicht-Akkreditierung
\item Akkreditierung mit Auflagen (in der Regel in 12 Monaten nachzuweisen) \emph{künftig nur noch als Ausnahme?!}
\item Erstmalige Akkreditierung: 8 Jahre
\item Reakkreditierung: 8 Jahre
\item früher gab es noch das \emph{Aussetzten des Verfahrens}
\end{itemize}
\end{frame}
%%%%%%%%%%%%%%%%%%%%%%%%%%%%%%%%%%%%%%%%%%%%%%%%%%%%%%%%%%%%%%%
\section{Mitgestaltung}
\frame{\tableofcontents[currentsection]}
%%%%%%%%%%%%%%%%%%%%%%%%%%%%%%%%%%%%%%%%%%%%%%%%%%%%%%%%%%%%%%%
\begin{frame}
\frametitle{Studentischer Akkreditierungspool}
Stellt studentische Vertretung:
\begin{itemize}
\item in Gutachtergruppen bei Programm- und Systemakkreditierung
\item im Akkreditierungsrat
\item als die Agenturen noch entschieden haben auch in Kommissionen dort
\end{itemize}
Organisiert:
\begin{itemize}
\item Vernetzungstreffen
\item \textbf{Schulungsseminare}
\end{itemize}
\hspace*{7cm}\includegraphics[scale=0.6]{pool.PNG}
\end{frame}
%%%%%%%%%%%%%%%%%%%%%%%%%%%%%%%%%%%%%%%%%%%%%%%%%%%%%%%%%%%%%%%
\begin{frame}
\frametitle{Studentischer Akkreditierungspool}
Mitglieder werden entsendet von:
\begin{itemize}
\item Bundesfachschaftentagungen
\item fzs (freier zusammenschluss von studentInnenschaften)
\item Landes-Asten-Konferenzen
\end{itemize}
\hspace*{7cm}\includegraphics[scale=0.6]{pool.PNG}
\end{frame}
%%%%%%%%%%%%%%%%%%%%%%%%%%%%%%%%%%%%%%%%%%%%%%%%%%%%%%%%%%%%%%%
\begin{frame}
\frametitle{Andere Orte der Mitwirkung}
An Euren Hochschulen:
\begin{itemize}
\item Gremien auf unterschiedlichen Ebenen: Qualitätszirkel, Akkreditierungskommissionen, Studienkommissionen, teilweise Prüfungskommissionen...
\item Erstellung des Selbstberichtes
\item Vor-Ort Begehung
\item evtl. Stellungnahme vor Akkreditierungsentscheidung
\end{itemize}
\vspace*{0.5cm}
Auf der BuFaTas usw. ...
\end{frame}
%%%%%%%%%%%%%%%%%%%%%%%%%%%%%%%%%%%%%%%%%%%%%%%%%%%%%%%%%%%%%%%
\section{Lehramts-\glqq spezifische\grqq\xspace Regelungen}

\begin{frame}
\frametitle{\insertsection}
\framesubtitle{Behandlung als Kombinationsstudiengang}
Das Lehramt ist typischerweise als Kombinationsstudiengang organisiert.
\begin{itemize}
	\item Kombinationsstudiengang: Mehrere F"acher sind w"ahlbar
	\item \S 32 MRVO Abs. 2: Akkreditierungsgegenstand ist der Kombinationsstudiengang. Die Hochschulen stellen durch ihr jeweiliges Qualitätsmanagement sicher, dass die Studierbarkeit in allen möglichen Fächerkombinationen gegeben ist. 
	\item besonderes Studiengangsprofil f"ur M Ed ist vorgeschrieben (MRVO \S 4)
\end{itemize}
\end{frame}

\begin{frame}
\frametitle{\insertsection}
\framesubtitle{Besondere Pr"ufbereiche}
\textit{\S 13 MRVO Abs. 3}\\\vspace*{.5cm}
Im Rahmen der Akkreditierung von Lehramtsstudiengängen ist 
insbesondere zu prüfen, ob 
\begin{enumerate}
\item   ein integratives Studium an Universitäten oder gleichgestellten Hochschulen von mindestens zwei Fachwissenschaften und von Bildungswissenschaften in der Bachelorphase sowie  in der Masterphase (Ausnahmen sind bei den Fächern Kunst und Musik zulässig), 
\item schulpraktische Studien bereits während des Bachelorstudiums und 
\item eine Differenzierung des Studiums und der Abschlüsse nach Lehrämtern
\end{enumerate}
 
erfolgt  sind. 
\end{frame}

%%%%%%%%%%%%%%%%%%%%%%%%%%%%%%%%%%%%%%%%%%%%%%%%%%%%%%%%%%%%%%%
\section*{Fragen}
\begin{frame}
\centering
\Large{ Fragen?}
\end{frame}
\end{document}
