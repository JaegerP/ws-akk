\documentclass{scrartcl}
\usepackage[ngerman]{babel}
\usepackage[utf8]{inputenc}
\usepackage{scrpage2}
\usepackage{graphicx}
\graphicspath{{./img/}}

\begin{document}
\pagestyle{scrheadings}
\clearscrheadfoot
\cfoot[]{\includegraphics[width=3cm]{CC-BY-SA.png}}
\section*{Deutsches Akkreditierungssystem}

§ 3 und § 9 des Studienakkreditierungsstaatsvertrags und § 24 und § 25 der Musterrechtsverordnung bestimmen die Akteure des deutschen Akkreditierungsystems und ihre Aufgaben. Beide sind auf der Startseite der Webseite des Akkreditierungsrates (www.akkreditierungsrat.de) zu finden. \\

\vspace{1cm}

\textbf{Lies zu erst die Kurzzusammenfassung ``Akkreditierungssystem`` (linke Navigationsleiste) auf der Seite des Akkreditierungsrates. Schaue dann auf die genannten § und beantworte folgende Fragen:}

\begin{itemize}
\item Welche Aufgaben hat der Akkreditierungsrat?
	\begin{itemize}
		\item Wer gehört ihm an?
	\end{itemize}
\item Welche Aufgabe haben die Akkreditierungsagenturen? 
	\begin{itemize}
		\item Welche Bedingungen muss eine Agentur erfüllen um in Deutschland akkreditieren zu dürfen?
		\item Was ist der European Quality Assurance Register for Higher Education (EQAR)?
	\end{itemize}
\item Welche Aufgabe haben die Gutachter? 
	\begin{itemize}
		\item Wie sieht eine Gutachtergruppe aus?
	\end{itemize}
\item Welche Aufgabe haben die Hochschulen im Akkreditierungssystem?
\end{itemize}

\textbf{Durchsuche die Dokumente nach folgenden Fragstellungen:}
\begin{itemize}
\item Welchen Einfluss hat die Kultusministerkonferenz (KMK)?
\item Welchen Einfluss hat die Hochschulrektorenkonferenz (HRK)?
\end{itemize}

\vspace{1cm}

\textbf{Skizziere ein Schaubild des deutschen Akkreditierungssystems mit seinen Akteuren: Akkreditierungsrat, Agentur, Gutachter, Hochschule, KMK und HRK. \\
Markiere an welchen Stellen Studierende mitwirken.}

\newpage
\section*{Akkreditierungsverfahren Deutschland}
Durch die Veränderungen, die mit Beginn des Jahres 2018 gültig geworden sind, wurde der Ablauf von Akkreditierungsverfahren geändert. Ein wichtiger Bestandteil bleibt allerdings die ''Begehung'', bei der die Gutachter die Hochschule besuchen und Vor-Ort Gespräche führen, die sie gemeinsam mit einem vorgelegten Bericht der Hochschule nutzen, um ein Gutachten zu erstellen.

\vspace{0.5cm}

\textbf{Führt ein Gespräch mit ein ZaPFikon, welches bereits Gutachtererfahrung hat, um mehr über die Begehung zu erfahren. Nutzt dazu folgende Fragen: }

\begin{itemize}
\item Teilnehmende/Akteure
	\begin{itemize}
		\item Welche Akteure gibt es?
		\item Was tun sie?
		\item Welche Akteure gehören zur Hochschule?
		\item Welche Akteure gehören zur Agentur?
		\item Welche Akteure sind unabhängig?
		\item Was passiert vor der Vor-Ort Begehung?
	\end{itemize}
\item Wie läuft eine Vor-Ort-Begehung ab?
\item Was sind die Aufgaben der Gutachtergruppe?
\item Wer erstellt wann welche Dokumente bzw. Berichte?
\item Was passiert im Anschluss an eine Begehung?
\end{itemize}

\textbf{Wie ordnet sich die Vor-Ort-Begehung in das allgemeine Verfahren ein? Siehe dazu § 3 des Studienakkreditierungsstaatsvertrags und Teil 4 (§ 22 bis § 25) der Musterrechtsverordnung (Beide sind auf der Startseite der Webseite des Akkreditierungsrates (www.akkreditierungsrat.de) zu finden).}

\newpage
\section*{Akkreditierungskriterien Deutschland}
Der Akkreditierungsrat veröffentlicht die Akkreditierungsberichte und -entscheidungen in einer Datenbank (www.akkreditierungsrat.de, linke Navigationsleite '' Akkreditierte Studiengänge und Hochschulen'', Link '' Zentrale Datenback akkreditierter Studiengänge nach altem Recht).\\

\vspace{0.5 cm}

\textbf{Sucht einen Eurer Studiengänge in der Datenbank und schaut Euch das Gutachten an. Falls es für diesen Studiengang eines gibt, ist es ganz unten auf der Seite des Studienganges verlinkt. Beantwortet die folgenden Fragen:}
\begin{itemize}
\item Wie ist das Gutachten aufgebaut?
\item Welche verschiedenen Kapitel gibt es?
\item Sind Beschreibung und Bewertung getrennt?
\item Wurde mehr als ein Studiengang bewertet? (Falls ja, sucht einen aus)
\end{itemize}
\vspace{0.5 cm}
\textbf{Erstellt eine Stichwortliste der geprüften Kriterien.\\}
Ignoriert die Auflagen und Empfehlungen in den Gutachten. Fürs Erste geht es nur um die Prüfkriterien.\\

\vspace{0.5 cm}
 \textbf{Vergleicht mit dem Raster für die Akkreditierungsberichte des Akkreditierungsrates (www.akkteditierungsrat.de, linke Navigationsleiste ''Kriterien und Verfahrensregeln'', Datei vom 29.03.2018.)\\
 Was sind formale Kriterien? Was sind fachlich-inhaltliche Kriterien?}



\newpage
\section*{Regeln für die Studiengangsgestaltung Deutschland}
Die Musterrechtsverordnung (auf der Startseite des Akkreditierungsrates www.akkreditierungsrat.de zu finden) löst die Ländergemeinsamen Strukturvorgaben in den Regelungen zur Studiengangsgestaltung ab.
Sie schreibt vor, dass Studiengänge modularisiert und studierbar sein müssen und führt das Leistungspunktesystem (ECTS) ein. (§ 7, § 8 und § 12).

\vspace{0.5 cm}

Beantworte folgende Fragen:
\begin{itemize}
	\item Was ist ein Modul?
	\item Über welchen Zeitraum erstreckt sich ein Modul?
	\item Welche Mindestgröße (ECTS-Leistungspunkte) soll es haben?
	\item Wie viele Prüfungen darf es enthalten?
	\item Was gehört in eine Modulbeschreibung?
	\item Welche ECTS Vorgaben gibt es für Bachelor- und Masterarbeit?
	\item Wie viele Arbeitsstunden entsprechen einem ECTS-Leistungspunkt?
	\item Wie viele ECTS-Leistungspunkte soll ein Semester umfassen?
	\item Wie viele ECTS-Leistungspunkte werden für den Bachelor, wie viele für den Master veranschlagt?
	\end{itemize}
	
\textbf{Sucht eine Modulbeschreibung eines Eurer Studiengänge (wo findet man sie?) und prüft, ob es den Kriterien, der Musterrahmenverordnung entspricht.}

\end{document}