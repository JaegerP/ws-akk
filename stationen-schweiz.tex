\documentclass{scrartcl}
\usepackage[german]{babel}
\usepackage[utf8]{inputenc}
\begin{document}
\section*{Schweizerisches Akkreditierungssystem}

Das Hochschulförderungs- und -koordinationsgesetz (HFKG) und die Verordnung des Hochschulrates über die Akkreditierung im Hochschulbereich regelt die Aufgaben und Zusammensetzung der Akteure im schweizerischen Akkreditierungssystem.

\vspace{1cm}

\textbf{Lies zu erst die Kurzzusammenfassung ``Akkreditierung Schweiz`` auf der Seite des Akkreditierungsrates (http://akkreditierungsrat.ch/akkreditierung-schweiz/).\\
Unten findest Du  unter ''Rechtliche  Grundlagen'' das HFKG und die Akkreditierungsverordnung HFKG. Schaue auf die Art. 21, 22 und Art. 28 - 35 im HFKG und Art. 10 - 13 in der Akkreditierungsverordnung und beantworte folgende Fragen:}

\begin{itemize}
\item Welche Aufgaben hat der Akkreditierungsrat?
	\begin{itemize}
		\item Wer gehört ihm an?
	\end{itemize}
\item Welche Arten von Akkreditierungen gibt es?
\begin{itemize}
\item Wofür ist eine institutionelle Akkreditierung Voraussetzung?
\end{itemize}
\item Welche Aufgabe hat die Schweizerische Akkreditierungsagentur (AAQ)? 
\item Welche Aufgabe haben die Gutachter? 
	\begin{itemize}
		\item Wie sieht eine Gutachtergruppe aus?
	\end{itemize}
\item Welche Aufgabe haben die Hochschulen im Akkreditierungssystem?
\end{itemize}


\vspace{1cm}

\textbf{Skizziere ein Schaubild des schweizerischen Akkreditierungssystems mit seinen Akteuren: Akkreditierungsrat, Agentur, Gutachter, Hochschule. \\
Markiere an welchen Stellen Studierende mitwirken.}


\newpage
\section*{Akkreditierungsverfahren Schweiz}
Abschnitt 5 der Akkreditierungsverordnung regelt den Ablauf einer erstmaligen Akkreditierung. Ein wichtiger Bestandteil ist die ''Begehung'' (Externe Begutachtung), bei der die Gutachter die Hochschule besuchen und Vor-Ort Gespräche führen, die sie gemeinsam mit einem vorgelegten Bericht der Hochschule nutzen, um ein Gutachten zu erstellen.

\vspace{0.5cm}

\textbf{Führt ein Gespräch mit ein ZaPFikon, welches bereits Gutachtererfahrung hat, um mehr über die Begehung zu erfahren. Nutzt dazu folgende Fragen: }

\begin{itemize}
\item Teilnehmende/Akteure
	\begin{itemize}
		\item Welche Akteure gibt es?
		\item Was tun sie?
		\item Welche Akteure gehören zur Hochschule?
		\item Welche Akteure gehören zur Agentur?
		\item Welche Akteure sind unabhängig?
		\item Was passiert vor der Vor-Ort Begehung?
	\end{itemize}
\item Wie läuft eine Vor-Ort-Begehung ab?
\item Was sind die Aufgaben der Gutachtergruppe?
\item Wer erstellt wann welche Dokumente bzw. Berichte?
\item Was passiert im Anschluss an eine Begehung?
\end{itemize}

\textbf{Wie ordnet sich die Vor-Ort-Begehung in das allgemeine Verfahren ein? Siehe dazu Abschnitt 5 der Akkreditierungsverordnung (Zu finden ganz unten auf der Seite des Akkreditierungsrates\\
http://akkreditierungsrat.ch/akkreditierung-schweiz/.)}

\newpage
\section*{Akkreditierungskriterien Schweiz}
Der Akkreditierungsrat veröffentlicht die Akkreditierungsentscheide auf seiner Webseite (http://akkreditierungsrat.ch/akkreditierungsentscheide/institutionelle-akkreditierung/).\\

\vspace{0.5 cm}
\textbf{Sucht Euch eine Hochschule aus und schaut Euch das Gutachten (Download PDF) an. Beantwortet die folgenden Fragen:}
\begin{itemize}
\item Wie ist der Schlussbericht aufgebaut?
\item Welche verschiedenen Kapitel gibt es?
\item Was sind die beurteilten Qualitätsstandards?
\item In welche Abschnitte ist die Beurteilung unterteilt?
\item Sind Beschreibung und Bewertung getrennt?
\end{itemize}
\vspace{0.5 cm}
\textbf{Erstellt eine Stichwortliste der beurteilten Qualitätsstandards.\\}
Welche davon sind relevant für die Lehre?
Ignoriert die Auflagen und Empfehlungen in den Gutachten.\\ 

\vspace{0.5 cm}
 \textbf{Vergleicht mit Anhang 1 der Akkreditierungsverordnung (Zu finden ganz unten auf der Seite des Akkreditierungsrates\\ http://akkreditierungsrat.ch/akkreditierung-schweiz/.)}

\newpage
\section*{Regeln für die Studiengangsgestaltung Schweiz}
Die Rektorenkonferenz der Schweizer Universitäten (CRUS) hat Empfehlungen für die Anwendung von ECTS basierende auf dem ECTS-User's Guide verabschiedet (www.swissuniversities.ch $\rightarrow$ Themen $\rightarrow$ Lehre $\rightarrow$ Bologna-Reform/ ECTS Link zum PDF im Text aktualisiert 2004). Zusätzlich wurden 2015 Richtlinien des Hochschulrates veröffentlicht (gleiche Seite, rechte Navigationsleiste '' Bologna-Richtlinien UH'').

\vspace{0.5 cm}

Beantworte mit Hilfe beider Dokumente folgende Fragen:
\begin{itemize}
	\item Was ist ein Modul?
	\item Über welchen Zeitraum erstreckt sich ein Modul?
	\item Welche Mindestgröße (ECTS-Leistungspunkte) muss es haben?
	\item Wie viele Prüfungen darf es enthalten?
	\item Was gehört in eine Modulbeschreibung?
	\item Welche ECTS Vorgaben gibt es für Bachelor- und Masterarbeit?
	\item Wie viele Arbeitsstunden entsprechen einem ECTS-Leistungspunkt?
	\item Wie viele ECTS-Leistungspunkte soll ein Semester umfassen?
	\item Wie viele ECTS-Leistungspunkte werden für den Bachelor, wie viele für den Master veranschlagt?
	\end{itemize}
	
\textbf{Sucht eine Modulbeschreibung eines Eurer Studiengänge (wo findet man sie?) und prüft, ob es den Kriterien, der Musterrahmenverordnung entspricht.}

\end{document}