\documentclass{scrartcl}
\usepackage[ngerman]{babel}
\usepackage[utf8]{inputenc}
\usepackage{booktabs}
\usepackage{graphicx}
\graphicspath{{./img/}}
\begin{document}
\thispagestyle{empty}
\section*{Persönlicher Merkzettel zum Akkreditierungs-Workshop}
%Nach einer allgemeinen Einführung darüber, was Akkreditierung ist, gibt es folgende Stationen, die in beliebiger Reihenfolge alle bearbeitet werden sollen. An jeder Station gibt es einen Zettel mit einem Arbeitsauftrag. Die Stationen sind:
%  	\begin{itemize}
%  		\item Deutsches Akkreditierungssystem
%  		\item Akkreditierungsverfahren
%  		\item Akkreditierungskriterien
%  		\item Regeln für die Studiengangsgestaltung
%  	\end{itemize}

Ziel des Workshops ist es, einen Überblick über die folgenden Fragen zu bekommen:
\vspace{0.5 cm}

\begin{tabular}{p{0.9\textwidth}}
Was sind die entscheidenden Dokumente für die Akkreditierung? Wo findet man Informationen? \\
\\
\hline\\
\hline\\
\hline\\

Wie ist das nationale Akkreditierungssystem aufgebaut?\\
\\
\\
\\
\\
\\
Wie sieht ein Akkreditierungsverfahren aus? Was für Verfahren gibt es?\\
\\
\\
\\
\\
\\
Was wird geprüft?\\
\\
\hline\\
\hline\\
\hline\\
Was für Akkreditierungsentscheidungen gibt es? Wie lange gelten sie?\\
\\
\hline\\
\hline\\
Wo können Studierende sich beteiligen?\\
\\
\hline\\
\hline\\
Was ist die Rolle der ZaPF?\\
\\
\hline\\
\hline\\
\hline\\

\end{tabular}\\
\hspace*{.8\linewidth}\includegraphics[width=3cm]{CC-BY-SA.png}

\end{document}