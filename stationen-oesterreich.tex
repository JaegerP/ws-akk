\documentclass{scrartcl}
\usepackage[ngerman]{babel}
\usepackage[utf8]{inputenc}
\usepackage{scrpage2}
\usepackage{graphicx}
\graphicspath{{./img/}}

\begin{document}
\pagestyle{scrheadings}
\clearscrheadfoot
\cfoot[]{\includegraphics[width=3cm]{CC-BY-SA.png}}
\section*{Österreichisches Akkreditierungssystem}

Das Hochschul-Qualitätssicherungsgesetz (HS-QSG) regelt die Qualitätssicherung an österreichischen Hochschulen.

\vspace{1cm}

\textbf{Lies zu erst die Kurzzusammenfassung ``Qualitätssicherung an Hochschulen in Österreich`` auf der Seite der Österreichische Hochschülerinnen- und Hochschülerschaft ÖH (https://qs.oeh.ac.at/qualitaetssicherung-hochschulen-oesterreich).\\
Unten findest Du  unter ''Rechtliche  Basis'' einen Link zum HS-QSG und den Auditrichtlinien AQ. Schaue auf § 2,3 des HS-QSG und und Kapitel 5 der Auditrichtlinien beantworte folgende Fragen:}

\begin{itemize}
\item Was für Akkreditierungen gibt es?
\item Welche Aufgaben hat die Agentur für Qualitätssicherung und Akkreditierung Austria?
	\begin{itemize}
		\item Wer gehört dem Board an?
		\item Was sind die Aufgaben des Boards?
	\end{itemize}
\item Welche Aufgabe haben die Gutachter? 
	\begin{itemize}
		\item Wie sieht eine Gutachtergruppe aus?
	\end{itemize}
\item Welche Aufgabe haben die Hochschulen im Akkreditierungssystem?
\end{itemize}


\vspace{1cm}

\textbf{Skizziere ein Schaubild des österreichischen Akkreditierungssystems mit seinen Akteuren: Agentur, Gutachter, Hochschule. \\
Markiere an welchen Stellen Studierende mitwirken.}


\newpage
\section*{Akkreditierungsverfahren Österreich}
Abschnitt 5 der Auditrichtlinien regelt den Ablauf einer erstmaligen Akkreditierung. Ein wichtiger Bestandteil ist die ''Begehung'' (Externe Begutachtung), bei der die Gutachter die Hochschule besuchen und Vor-Ort Gespräche führen, die sie gemeinsam mit einem vorgelegten Bericht der Hochschule nutzen, um ein Gutachten zu erstellen.

\vspace{0.5cm}

\textbf{Führt ein Gespräch mit ein ZaPFikon, welches bereits Gutachtererfahrung hat, um mehr über die Begehung zu erfahren. Nutzt dazu folgende Fragen: }

\begin{itemize}
\item Teilnehmende/Akteure
	\begin{itemize}
		\item Welche Akteure gibt es?
		\item Was tun sie?
		\item Welche Akteure gehören zur Hochschule?
		\item Welche Akteure gehören zur Agentur?
		\item Welche Akteure sind unabhängig?
		\item Was passiert vor der Vor-Ort Begehung?
	\end{itemize}
\item Wie läuft eine Vor-Ort-Begehung ab?
\item Was sind die Aufgaben der Gutachtergruppe?
\item Wer erstellt wann welche Dokumente bzw. Berichte?
\item Was passiert im Anschluss an eine Begehung?
\end{itemize}

\textbf{Wie ordnet sich die Vor-Ort-Begehung in das allgemeine Verfahren ein? Siehe dazu Abschnitt 5 der Auditrichtlinien (Zu finden ganz unten auf der Seite der Österreichische Hochschülerinnen- und Hochschülerschaft ÖH\\
https://qs.oeh.ac.at/qualitaetssicherung-hochschulen-oesterreich.)}

\newpage
\section*{Akkreditierungskriterien Österreich}
Die Agentur für Qualitätssicherung und Akkreditierung Austria veröffentlicht ihre Akkreditierungs- und Zertifizierungsentscheidungen (https://www.aq.ac.at/de/akkreditierte-hochschulen-studien/).

\vspace{0.5 cm}
\textbf{Sucht Euch eine Hochschule aus und schaut Euch das Gutachten an (Vorsicht nicht mit Ergebnisbericht, Stellungnahme oder Erfüllung der Auflagen verwechseln). Beantwortet die folgenden Fragen:}
\begin{itemize}
\item Wie ist das Gutachten aufgebaut?
\item Welche verschiedenen Kapitel gibt es?
\item Was sind die beurteilten Qualitätsstandards?
\item In welche Abschnitte ist die Beurteilung unterteilt?
\item Sind Beschreibung und Bewertung getrennt?

\end{itemize}
\vspace{0.5 cm}
\textbf{Erstellt eine Stichwortliste der beurteilten Qualitätsstandards.\\}
Welche davon sind relevant für die Lehre?
Ignoriert die Auflagen und Empfehlungen in den Gutachten.\\ 

\vspace{0.5 cm}
 \textbf{Vergleicht mit § 22 des HS-QSG und mit Kapitel 4 der Auditrichtlinien (Zu finden ganz unten auf der Seite der Österreichische Hochschülerinnen- und Hochschülerschaft ÖH\\
https://qs.oeh.ac.at/qualitaetssicherung-hochschulen-oesterreich.)}

\newpage
\section*{Regeln für die Studiengangsgestaltung Österreich}
Die Agentur für Qualitätssicherung und Akkreditierung Austria hat Empfehlungen der Österreichischen Bologna Follow-Up Gruppe zur Umsetzung des ECTS-Leitfadens veröffentlicht (www.aq.ac.at $\rightarrow$ Akkreditierung  $\rightarrow$ Akkreditierung in Österreich  $\rightarrow$ Fachhochschulen  $\rightarrow$ Downloads  $\rightarrow$ PDF ziemlich weit unten). Hier bei handelt es sich um Empfehlungen für die konkrete Umsetzung des europäischen ECTS User's Guide.
\vspace{0.5 cm}

Beantworte mit Hilfe der Empfehlungen folgende Fragen:
\begin{itemize}
	\item Was ist ein Modul?
	\item Über welchen Zeitpunkt erstreckt sich ein Modul?
	\item Welche Mindestgröße (ECTS-Leistungspunkte) muss es haben?
	\item Wie viele Prüfungen darf es enthalten?
	\item Was gehört in eine Modulbeschreibung?
	\item Welche ECTS Vorgaben gibt es für Bachelor- und Masterarbeit?
	\item Wie viele Arbeitsstunden entsprechen einem ECTS-Leistungspunkt?
	\item Wie viele ECTS-Leistungspunkte soll ein Semester umfassen?
	\item Wie viele ECTS-Leistungspunkte werden für den Bachelor, wie viele für den Master veranschlagt?
	\end{itemize}
	
\textbf{Sucht eine Modulbeschreibung eines Eurer Studiengänge (wo findet man sie?) und prüft, ob es den Kriterien, der Musterrahmenverordnung entspricht.}

\end{document}